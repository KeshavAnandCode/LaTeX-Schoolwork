\documentclass[12pt,letterpaper]{article}

\usepackage{graphicx} % For images
\usepackage{titlesec}
\usepackage{tabularx}
\usepackage{amsmath}
\usepackage[colorlinks=true, linkcolor=black, urlcolor=black, citecolor=black]{hyperref}

\usepackage[T1]{fontenc}  % ensures proper font encoding
\usepackage{textcomp}      % provides \texteuro
\usepackage[font=small,labelfont=bf]{caption}
\usepackage{booktabs}   % in your preamble
\renewcommand{\arraystretch}{1.2} % row height
\usepackage{subcaption}   % add in preamble


% ---------- Word-style margins ----------
\setlength{\oddsidemargin}{0in}  
\setlength{\evensidemargin}{0in} 
\setlength{\textwidth}{6.5in}    
\setlength{\topmargin}{-0.5in}      
\setlength{\textheight}{9in} 

\setlength{\parindent}{0pt}


\titleformat{\section} % top-level section
  {\Large\bfseries\scshape} % Large font, bold, small caps
  {\thesection} % shows 1, 2, 3, ...
  {1em} % spacing between number and title
  {}

\titleformat{\subsection}
  {\large\bfseries\scshape}
  {\thesubsection}
  {1.5em} % slightly more spacing
  {}

\titleformat{\subsubsection}
  {\normalsize\bfseries}
  {\thesubsubsection}
  {2em} % more spacing for deeper hierarchy
  {}

\begin{document}



% ---------- SIMPLE HEADER ----------
\pagenumbering{arabic}      % start with Arabic page numbers
\setcounter{page}{1}

% Header: title centered, partner left, date right
\begin{center}
    \textbf{\large Using an Inclined Plane to Determine Kinetic and Static Frictional Coefficient of a Wooden Block on a Metal Track}
\end{center}

\vspace{1em}

\noindent
\textbf{Lab Partners:} Jose P, Dheveshvar S., Amrutha V.
\hfill
\textbf{Date:} 29 October 2025
% ---------- MAIN DOCUMENT ----------

% Acknowledgment
\section*{Objective}
This lab's objective is to determine the coefficient of static friction between an adjustable inclined plane and a block.

\section*{Introduction}
Friction is a force that opposes the relative motion of two surfaces in contact.
The frictional force is directly proportional to the normal force between the surfaces and is characterized by the coefficient of friction, denoted by $\mu$.'
There are two main types of friction: static friction, which acts when the surfaces are at rest relative to each other, and kinetic friction, which acts when the surfaces are in motion relative to each other.
It is important to note that the static frictional force is equal to the remaining forces in the opposing direction until it reaches its maximum value, which is given by $F_{f_{s}} = \mu_s N$, where $\mu_s$ is the coefficient of static friction and $N$ is the normal force.
The kinetic frictional force is given by $F_{s_{k}} = \mu_k N$, where $\mu_k$ is the coefficient of kinetic friction. These coefficients can only be experimentally determined and are unique to the material 
and composition of the two surfaces in contact. Hence, this experiment aims to use an inclined plane to deteerm the coefficients of static and kinetic friction between a wooden block and a metal track.

\section*{Materials}

The following methods and apparatus were used to determine the coefficients of static and kinetic friction:\\
\begin{itemize}
    \item 119g Wooden block
    \item Adjustable angle metal inclined plane with protractor
\end{itemize}


\section*{Procedure}
\begin{enumerate}
    \item Set the plane angle to $0^\circ$ and place the block at the far end, roughly 10 cm from the edge.
    \item Slowly raise the plane and stop when the block starts to slide down.
    \item Record the angle value in a data table under \textbf{static friction}.
    \item Repeat steps 1--3 five times, then take the average of the angles. This average angle will be used to calculate static friction.
    \item Repeat step 1.
    \item Slowly raise the plane while tapping the edge to overcome the static friction. Stop when the block starts to slide down the ramp without slowing down.
    \item Record the angle value in a data table under \textbf{kinetic friction}.
    \item Repeat steps 5--7 five times, then take the average of the angles for kinetic friction.
\end{enumerate}

\section*{Experimental Setup}

\begin{figure}[h!] % h! = “here” placement
    \centering
    \includegraphics[width=0.7\textwidth]{Sketch} % <-- your image file name
    \caption{Experimental setup for measuring static and kinetic friction using an inclined plane.}
    \label{fig:friction_setup}
\end{figure}

\newpage

\section*{Results}

\begin{table}[h!]
\centering
\caption{Measured critical angles for static and kinetic friction.}
\label{tab:friction_angles}
\begin{tabularx}{0.8\textwidth}{@{}lcc@{}}
\toprule
\textbf{Trial} & \textbf{Static Friction Angle ($^\circ$)} & \textbf{Kinetic Friction Angle ($^\circ$)} \\
\midrule
1 & 19.0 & 15.0 \\
2 & 17.5 & 14.5 \\
3 & 18.5 & 15.0 \\
4 & 18.0 & 16.0 \\
5 & 19.0 & 14.0 \\
\midrule
\textbf{Average} & 18.4 & 14.9 \\
\bottomrule
\end{tabularx}
\end{table}



The coefficients of static and kinetic friction were calculated using the relationship:
\[
\mu = \tan(\theta)
\]
where $\theta$ is the average critical angle measured for each case.


Average angle for static friction:
\[
\theta_s = 18.4^\circ
\]

Coefficient of static friction:
\[
\mu_s = \tan(18.4^\circ) = 0.33
\]


Average angle for kinetic friction:
\[
\theta_k = 14.9^\circ
\]

Coefficient of kinetic friction:
\[
\mu_k = \tan(14.9^\circ) = 0.27
\]
\\\\
Note that the minute acceleration due to the tapping of the block is neglected, as it's acceleration is assumed to be negligible. Also neglected is the force of air resistance, as it is assumed to be very small compared to the other forces acting on the block.

\section*{Error Analysis}

The first major source of error lies in the method of raising the angle to the right amount. The acceleration of the raising of the block must be zero; otherwise,
the block may start to slide earlier than it should have. Hence, the raising of the block is assumed to be the major source of error in this experiment.
Alongside this, the human reaction time to accurately read the protractor within decent tolerance is also a source of error.
Since the coefficients of static and kinetic friction were calculated using $\tan{\theta}$ and involved precise decimal values, 
inconsistency in the measured angles could result in either an overestimate or underestimate of the actual coefficients.

\section*{Discussion}

The physics concepts used in the lab are the coefficients of static and kinetic friction, which have major applications in the real world, specifically in materials science and engineering.
For example, understanding friction is crucial in designing systems like car wheels, where controlling the friction is vital to driving in different conditions (especially slippery roads).
Another such application of friction is in the design of screws, as the friction within threads is what allows them to hold materials together securely. Hence, a lower coefficient
of static friction would result in a looser screw, while a higher coefficient would result in a tighter screw, which can be crucial in construction, manufacturing, and architecture.
Understanding static and kinetic friction is not only important in product design but can be crucial to ensure safety in various applications, impacting everyday life. \\\\
Another way that this lab can be carried out is by using a spring scale to directly meassure the force needed to start moving the block and keep it moving at a constant velocity.
This method would also accurate measure the coefficients of static and kinetic friction, although it may be prone to different sources of error. A second method this lab can be carried out
is by using a motion sensor to track the block's movement down the incline, and similar to the cart and ramp lab, the motion sensor can be used to derive the acceleration (by differentiating the velocity data).
From the acceleration, the net force acting on the block can be calculated using Newton's Second Law, which can be used to calculate the frictional coefficients.

\section*{Conclusion}



Through this lab, it was determined that the coefficient of static friction ($\mu_s$) of wood on metal is roughly \textbf{0.33}, and the coefficient of kinetic friction ($\mu_k$) is roughly \textbf{0.27}. 
The coefficient of static friction can be compared to the established range $0.2-0.6$ for such surfaces.
The experimentally determined value falls within this range, indicating that the results are accurate and that the lab was completed successfully.
\end{document}
