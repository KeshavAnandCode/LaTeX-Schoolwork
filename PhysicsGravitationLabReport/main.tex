\documentclass[12pt,letterpaper]{article}

\usepackage{graphicx} % For images
\usepackage{titlesec}
\usepackage{tabularx}
\usepackage{amsmath}
\usepackage[colorlinks=true, linkcolor=black, urlcolor=black, citecolor=black]{hyperref}

\usepackage[T1]{fontenc}  % ensures proper font encoding
\usepackage{textcomp}      % provides \texteuro
\usepackage[font=small,labelfont=bf]{caption}
\usepackage{booktabs}   % in your preamble
\renewcommand{\arraystretch}{1.2} % row height
\usepackage{subcaption}   % add in preamble

\usepackage{placeins}
\usepackage{float}

\graphicspath{{images/}}


% ---------- Word-style margins ----------
\setlength{\oddsidemargin}{0in}  
\setlength{\evensidemargin}{0in} 
\setlength{\textwidth}{6.5in}    
\setlength{\topmargin}{-0.5in}      
\setlength{\textheight}{9in} 

\setlength{\parindent}{0pt}


\titleformat{\section} % top-level section
  {\LARGE\bfseries\scshape} % Large font, bold, small caps
  {\thesection} % shows 1, 2, 3, ...
  {1em} % spacing between number and title
  {}

\titleformat{\subsection}
  {\large\bfseries\scshape}
  {\thesubsection}
  {1.5em} % slightly more spacing
  {}

\titleformat{\subsubsection}
  {\normalsize\bfseries}
  {\thesubsubsection}
  {2em} % more spacing for deeper hierarchy
  {}

\begin{document}



% ---------- SIMPLE HEADER ----------
\pagenumbering{arabic}      % start with Arabic page numbers
\setcounter{page}{1}

% Header: title centered, partner left, date right
\begin{center}
    \textbf{\large Experimentally Deriving Newton’s Universal Law of Gravitation by Measuring Mass, Distance, and G in a Simulation}
\end{center}

\vspace{1em}

\noindent
\textbf{Name:} Keshav Anand
\hfill
\textbf{Date:} 16 November 2025
% ---------- MAIN DOCUMENT ----------


\section*{Objectives}
Use the \href{http://phet.colorado.edu/sims/html/gravity-force-lab/latest/gravity-force-lab_en.html}{Gravitational force simulation} to determine the dependence of the gravitational force on the mass of the objects involved.\\

Use the same simulation to determine the dependence of the gravitational force on the distance between the two masses.\\

Determine the experimental value of the universal gravitational constant (G). (This is what relates the gravitational force to the masses and distance rather than being these proportional. G must be included in your final equation.)\\

Determine an Equation for the Universal Law of Gravitation based on your data, using only symbols.

\section*{Introduction}

\subsection*{Background}
The concept of gravitational force has been misunderstood for most of history,
commonly attributed to divine or supernatural causes until Sir Isaac Newton published his groundbreaking \textit{Principia}.
In the Principia, Newton asserted that every mass exerts an attractive force on every other mass,
 a phenomenon described by Newton's Universal Law of Gravitation (NLUG). 
This law states that the magnitude of the gravitational force between two masses is

\begin{equation}
F_g = G \frac{m_1 m_2}{r^2}
\label{eq:NLUG}
\end{equation}

where \(F\) is the gravitational force, \(m_1\) and \(m_2\) are the interacting masses, \(r\) is the distance between their centers.
\(G\) is the universal gravitational constant, a constant of proprtionality that has been calculated to be 
\[
G = 6.674 \times 10^{-11}\ \text{N}\,\text{m}^2\,\text{kg}^{-2}.
\]

In the scientific community, NLUG is treated as an absolute truth, and many important discoveries and applications
rely on its accuracy. From engineering to astrophysics, NLUG has profound importance, and its validity
is vital to the functioning of scientific advancement. 

\subsection*{Purpose}

This lab aims to use a computer simulation 
to verify NLUG by deriving the relationship between gravitational force, masses of objects, and the distance
between them. First, a gravitation simulation will be used to derive the relationship between objects'
masses and the gravitational force. Then, the same process will be repeated with comparing gravitational
force to the distance between objects. Data collected from these two setups will be used to (hopefully) re-establish
the relationship of proportionality proposed by Newton, and the collected data will also solve for the 
universal Gravitational constant, $G$.


\subsection*{Research Problem}
The research objective for this project is to verify the gravitational relationship between two objects and verify the
constant $G$. The primary problem of investigation is that theoretical mathematics often fails to adequately capture
a true relationship in the real world. Moreover, using physical objects and tools of measurements can result in
unwanted noisy data and is limited by the precision of measurement. Ergo, a simulation bridges this gap, 
allowing for an accurate verification of NLUG.




\section*{Methodology}
\subsection*{Materials and Resources}

As this lab was performed within a simulation, all physical materials are limited to a computer with at least 400 MB of memory to render the simulation.

Within the simulation, the simulated materials include
\begin{itemize}
    \item Adjustable Mass, $m_1$
    \item Adjustable Mass, $m_2$
    \item 10 Meter Scale
    \item Automatic Force Scale to Measure Gravitational Attraction
    \item Two simulated people holding $m_1$ and $m_2$ from colliding into each other due to gravitation
\end{itemize}


\subsection*{Experimental Setup}

\begin{figure}[h!] % h! = “here” placement
    \centering
    \includegraphics[width=0.7\textwidth]{ExperimentalSetup} % <-- your image file name
    \caption{Experimental setup for the gravity simulation}
    \label{fig:friction_setup}
\end{figure}

Note that in Figure \ref{fig:friction_setup}, all inputs (independent variables) are denoted in blue, whereas outputs (dependent variables) are denoted in red


\subsection*{Procedure}
\begin{enumerate}
    \item Set the location of mass 1 to exactly 2 meters on the scale, and set mass 2 to exactly 6 meters on the scale, with a distance between of 4 meters
    \item Set the mass of objects 1 and 2 to exactly 100 kg
    \item Set the force values to scientific notation, and uncheck the option for masses of constant size
    \item Leaving the mass of $m_2$ constant, change the mass of $m_1$ to be the values listed below, and record both the force on $m_1$ by $m_2$ and the force on $m_2$ by $m_1$
      \subitem Mass values for $m_1$ (kg): 50, 100, 250, 500, 750, 1000
    \item Reset the simulation as detailed by steps 1-3
    \item Leaving the mass of $m_1$ constant, change the mass of $m_2$ to be the values listed below, and record both the force on $m_1$ by $m_2$ and the force on $m_2$ by $m_1$
      \subitem Mass values for $m_2$ (kg): 50, 100, 250, 500, 750, 1000
    \item Reset the simulation as detailed by steps 1-3
    \item Change the masses of both $m_1$ and $m_2$ to be the values listed below, and record both the force on $m_1$ by $m_2$ and the force on $m_2$ by $m_1$
      \subitem Mass values for $m_1$ and $m_2$ (kg): 50, 100, 250, 500, 750, 1000
    \item Reset the simulation as detailed by steps 1-3
    \item Leave $m_2$ at 10 meters on the scale (align the black dot for center of mass), and move $m_1$ based on its center to the below values on the scale, and record both the force on $m_1$ by $m_2$ and the force on $m_2$ by $m_1$
    \subitem Position values for $m_1$ (m): 0, 2, 4, 6, 8
    \item Reset the simulation as detailed by steps 1-3
    \item Leave $m_1$ at 0 meters on the scale (align the black dot for center of mass), and move $m_2$ based on its center to the below values on the scale, and record both the force on $m_1$ by $m_2$ and the force on $m_2$ by $m_1$
    \subitem Position values for $m_1$ (m): 10, 8, 6, 4, 2

\end{enumerate}

Note that the above steps require the following raw data the be collected at each datapoint

\begin{itemize}
  \item Position of the center of mass of $m_1$, (m)
  \item Position of the center of mass of $m_2$, (m)
  \item Mass of $m_1$, (kg)
  \item Mass of $m_2$, (kg)
  \item Force on $m_1$ by $m_2$, (N)
  \item Force on $m_2$ by $m_1$, (N)
\end{itemize}

\section*{Results}



\subsection*{Raw Data}

While all data was collected jointly, the five separate experimental setups can be split up into the following tables for convenience:



\begin{table}[H]
\centering
\caption{Force between two masses while varying $m_1$.}
\label{tab:grav_m1}

% Increase horizontal space between columns

% Slightly increase vertical spacing
\renewcommand{\arraystretch}{1.3}

\begin{tabularx}{0.95\textwidth}{
    @{}
    >{\centering\arraybackslash}X
    >{\centering\arraybackslash}X
    >{\centering\arraybackslash}X
    >{\centering\arraybackslash}X
    >{\centering\arraybackslash}X
    >{\centering\arraybackslash}X
    >{\centering\arraybackslash}X
    @{}
}
\toprule
\textbf{Trial} &
\textbf{$m_1$ (kg)} &
\textbf{$m_2$ (kg)} &
\textbf{$x_1$ (m)} &
\textbf{$x_2$ (m)} &
\textbf{$F_{1\rightarrow2}$ (N)} &
\textbf{$F_{2\rightarrow1}$ (N)} \\
\midrule
1 & 50 & 100 & 2.00 & 6.00 & $2.09\times10^{-8}$ & $2.09\times10^{-8}$ \\
2 & 100 & 100 & 2.00 & 6.00 & $4.17\times10^{-8}$ & $4.17\times10^{-8}$ \\
3 & 250 & 100 & 2.00 & 6.00 & $1.04\times10^{-7}$ & $1.04\times10^{-7}$ \\
4 & 500 & 100 & 2.00 & 6.00 & $2.09\times10^{-7}$ & $2.09\times10^{-7}$ \\
5 & 750 & 100 & 2.00 & 6.00 & $3.13\times10^{-7}$ & $3.13\times10^{-7}$ \\
6 & 1000 & 100 & 2.00 & 6.00 & $4.17\times10^{-7}$ & $4.17\times10^{-7}$ \\
\bottomrule
\end{tabularx}
\end{table}

\begin{table}[H]
\centering
\caption{Force between two masses while varying $m_2$.}
\label{tab:grav_m2}
\renewcommand{\arraystretch}{1.3}

\begin{tabularx}{0.95\textwidth}{
    @{}
    >{\centering\arraybackslash}X
    >{\centering\arraybackslash}X
    >{\centering\arraybackslash}X
    >{\centering\arraybackslash}X
    >{\centering\arraybackslash}X
    >{\centering\arraybackslash}X
    >{\centering\arraybackslash}X
    @{}
}
\toprule
\textbf{Trial} &
\textbf{$m_1$ (kg)} &
\textbf{$m_2$ (kg)} &
\textbf{$x_1$ (m)} &
\textbf{$x_2$ (m)} &
\textbf{$F_{1\rightarrow2}$ (N)} &
\textbf{$F_{2\rightarrow1}$ (N)} \\
\midrule
1 & 100 & 50 & 2.00 & 6.00 & $2.09\times10^{-8}$ & $2.09\times10^{-8}$ \\
2 & 100 & 100 & 2.00 & 6.00 & $4.17\times10^{-8}$ & $4.17\times10^{-8}$ \\
3 & 100 & 250 & 2.00 & 6.00 & $1.04\times10^{-7}$ & $1.04\times10^{-7}$ \\
4 & 100 & 500 & 2.00 & 6.00 & $2.09\times10^{-7}$ & $2.09\times10^{-7}$ \\
5 & 100 & 750 & 2.00 & 6.00 & $3.13\times10^{-7}$ & $3.13\times10^{-7}$ \\
6 & 100 & 1000 & 2.00 & 6.00 & $4.17\times10^{-7}$ & $4.17\times10^{-7}$ \\
\bottomrule
\end{tabularx}
\end{table}

\begin{table}[H]
\centering
\caption{Force between equal masses while varying $m_1 = m_2$.}
\label{tab:grav_equal}
\renewcommand{\arraystretch}{1.3}

\begin{tabularx}{0.95\textwidth}{
    @{}
    >{\centering\arraybackslash}X
    >{\centering\arraybackslash}X
    >{\centering\arraybackslash}X
    >{\centering\arraybackslash}X
    >{\centering\arraybackslash}X
    >{\centering\arraybackslash}X
    >{\centering\arraybackslash}X
    @{}
}
\toprule
\textbf{Trial} &
\textbf{$m_1$ (kg)} &
\textbf{$m_2$ (kg)} &
\textbf{$x_1$ (m)} &
\textbf{$x_2$ (m)} &
\textbf{$F_{1\rightarrow2}$ (N)} &
\textbf{$F_{2\rightarrow1}$ (N)} \\
\midrule
1 & 50 & 50 & 2.00 & 6.00 & $1.04\times10^{-8}$ & $1.04\times10^{-8}$ \\
2 & 100 & 100 & 2.00 & 6.00 & $4.17\times10^{-8}$ & $4.17\times10^{-8}$ \\
3 & 250 & 250 & 2.00 & 6.00 & $2.61\times10^{-7}$ & $2.61\times10^{-7}$ \\
4 & 500 & 500 & 2.00 & 6.00 & $1.04\times10^{-6}$ & $1.04\times10^{-6}$ \\
5 & 750 & 750 & 2.00 & 6.00 & $2.35\times10^{-6}$ & $2.35\times10^{-6}$ \\
6 & 1000 & 1000 & 2.00 & 6.00 & $4.17\times10^{-6}$ & $4.17\times10^{-6}$ \\
\bottomrule
\end{tabularx}
\end{table}

\begin{table}[H]
\centering
\caption{Force between two masses while varying $x_1$.}
\label{tab:grav_dist1}
\renewcommand{\arraystretch}{1.3}

\begin{tabularx}{0.95\textwidth}{
    @{}
    >{\centering\arraybackslash}X
    >{\centering\arraybackslash}X
    >{\centering\arraybackslash}X
    >{\centering\arraybackslash}X
    >{\centering\arraybackslash}X
    >{\centering\arraybackslash}X
    >{\centering\arraybackslash}X
    @{}
}
\toprule
\textbf{Trial} &
\textbf{$m_1$ (kg)} &
\textbf{$m_2$ (kg)} &
\textbf{$x_1$ (m)} &
\textbf{$x_2$ (m)} &
\textbf{$F_{1\rightarrow2}$ (N)} &
\textbf{$F_{2\rightarrow1}$ (N)} \\
\midrule
1 & 100 & 100 & 0.00 & 10.00 & $6.67\times10^{-9}$ & $6.67\times10^{-9}$ \\
2 & 100 & 100 & 2.00 & 10.00 & $1.04\times10^{-8}$ & $1.04\times10^{-8}$ \\
3 & 100 & 100 & 4.00 & 10.00 & $1.85\times10^{-8}$ & $1.85\times10^{-8}$ \\
4 & 100 & 100 & 6.00 & 10.00 & $4.17\times10^{-8}$ & $4.17\times10^{-8}$ \\
5 & 100 & 100 & 8.00 & 10.00 & $1.67\times10^{-7}$ & $1.67\times10^{-7}$ \\
\bottomrule
\end{tabularx}
\end{table}

\begin{table}[H]
\centering
\caption{Force between two masses while varying $x_2$.}
\label{tab:grav_dist2}
\renewcommand{\arraystretch}{1.3}

\begin{tabularx}{0.95\textwidth}{
    @{}
    >{\centering\arraybackslash}X
    >{\centering\arraybackslash}X
    >{\centering\arraybackslash}X
    >{\centering\arraybackslash}X
    >{\centering\arraybackslash}X
    >{\centering\arraybackslash}X
    >{\centering\arraybackslash}X
    @{}
}
\toprule
\textbf{Trial} &
\textbf{$m_1$ (kg)} &
\textbf{$m_2$ (kg)} &
\textbf{$x_1$ (m)} &
\textbf{$x_2$ (m)} &
\textbf{$F_{1\rightarrow2}$ (N)} &
\textbf{$F_{2\rightarrow1}$ (N)} \\
\midrule
1 & 100 & 100 & 0.00 & 10.00 & $6.67\times10^{-9}$ & $6.67\times10^{-9}$ \\
2 & 100 & 100 & 0.00 & 8.00 & $1.04\times10^{-8}$ & $1.04\times10^{-8}$ \\
3 & 100 & 100 & 0.00 & 6.00 & $1.85\times10^{-8}$ & $1.85\times10^{-8}$ \\
4 & 100 & 100 & 0.00 & 4.00 & $4.17\times10^{-8}$ & $4.17\times10^{-8}$ \\
5 & 100 & 100 & 0.00 & 2.00 & $1.67\times10^{-7}$ & $1.67\times10^{-7}$ \\
\bottomrule
\end{tabularx}
\end{table}

\subsection*{Analysis}

The first major observation that can be made using the raw data is that for all data points (in all 5 tables), 
\textbf{$F_{1\rightarrow2}$ (N)} = \textbf{$F_{2\rightarrow1}$ (N)}. This observation is \textbf{Newton's Third Law},
as the force of one mass on another is equal to the other mass on it. Hence, these two forces can be replaced by one 
force column, denoted as $F_g$, which is the force of gravitational attraction between the two objects. Also notes that 
the direction of these two forces are always towards each others, which is why gravitation is known as a force of attraction.
\\\\
Then, look to Table \ref{tab:grav_m1}. Observe that there is a relationship between the changing variable ($m_1$)
and the gravitational force. As the other variables are all constant in this experiment, the relationship between 
$m_1$ and $F_g$ can be graphed as follows.



\begin{figure}[h!] % h! = “here” placement
    \centering
    \includegraphics[width=0.7\textwidth]{Force vs m1} % <-- your image file name
    \caption{$F_g$ vs $m_1$ graphed}
    \label{fig:m1graph}
\end{figure}


Observing Figure \ref{fig:m1graph}, there is a linear trend between the gravitational force and the mass of object 1.
This can be represented by the proportionality:
\[
F_g \propto m_1.
\]
Hence, the relationship can be described with the below equation, where $k_1$ is simply a constant:

\begin{equation}
F_g = k_1 \times m_1
\label{eq:m1}
\end{equation}

Applying the same logic to Table \ref{tab:grav_m2} yields the following relationship between $F_g$ and $m_2$


\begin{figure}[h!] % h! = “here” placement
    \centering
    \includegraphics[width=0.7\textwidth]{Force vs m2} % <-- your image file name
    \caption{$F_g$ vs $m_2$ graphed}
    \label{fig:m2graph}
\end{figure}

Note that again, there is a linear trend, implying 

\[
F_g \propto m_2.
\]

Hence, the relationship can be described with the below equation, where $k_2$ is simply another constant:

\begin{equation}
F_g = k_2 \times m_2.
\label{eq:m2}
\end{equation}

Looking at equation \ref{eq:m1} and \ref{eq:m2}, they can be safely combined into one equation, where there is yet another constant of proportionality $k_3$:

\begin{equation}
F_g = k_3 (m_1 \times m_2).
\label{eq:m1and2}
\end{equation}

Note, that when $m_1$ and $m_2$ are the same, equation \ref{eq:m1and2} simplifies to 

\[
F_g = k_3  \times m^2.
\]

Looking at Table \ref{tab:grav_equal}, this derived relationship can be verified by graphing as follows:



\begin{figure}[H] % h! = “here” placement
    \centering
    \includegraphics[width=0.7\textwidth]{Force vs Mass m1 m2} % <-- your image file name
    \caption{$F_g$ vs mass of $m_1$ = $m_2$}
    \label{fig:m1m2graph}
\end{figure}


As suggested by equation \ref{eq:m1and2}, there is a proportional quadratic relationship between the masses of the objects and the resulting gravitational force (see Figure \ref{fig:m1m2graph}). 
Therefore, equation \ref{eq:m1and2} is validated by the simulation.\\\\

Moving onto Tables \ref{tab:grav_dist1} and \ref{tab:grav_dist2}, the output columns for the gravitational force appear identical. 
Upon further inspection, the gravitational force seems to exhibit a dependency on the distance between the masses.
A processed data table can be made combining Tables \ref{tab:grav_dist1} and \ref{tab:grav_dist2},
showcasing the relationship between $F_g$ and $r:= |x_1-x_2|$.

\begin{table}[H]
\centering
\caption{Gravitational force versus distance between two masses.}
\label{tab:grav_vs_r}
\renewcommand{\arraystretch}{1.3}

\begin{tabularx}{0.8\textwidth}{
    @{}
    >{\centering\arraybackslash}X
    >{\centering\arraybackslash}X
    @{}
}
\toprule
\textbf{Distance $r$ (m)} & \textbf{Gravitational Force $F_g$ (N)} \\
\midrule
2.00 & $1.67\times10^{-7}$ \\
4.00 & $4.17\times10^{-8}$ \\
6.00 & $1.85\times10^{-8}$ \\
8.00 & $1.04\times10^{-8}$ \\
10.00 & $6.67\times10^{-9}$ \\
\bottomrule
\end{tabularx}
\end{table}

This new relationship is graphed in the figure below:


\begin{figure}[H] % h! = “here” placement
    \centering
    \includegraphics[width=0.7\textwidth]{Force vs Distance Between Masses} % <-- your image file name
    \caption{$F_g$ vs $r= x_2 - x_1$}
    \label{fig:rgraph}
\end{figure}

The shown trendline suggests a proprtional fit to $\frac{1}{r^2}$ as follows:

\[
F_g \propto \frac{1}{r^2},
\]

so 

\begin{equation}
F_g = k_4 \times \frac{1}{r^2}.
\label{eq:finalEq}
\end{equation}

When combined with equation \ref{eq:m1and2}, NLUG pops out, and the proprtionality constant can be denotes as $G$:


\begin{equation}
F_g = G \frac{m_1 m_2}{r^2} \tag{\ref{eq:NLUG}}
\end{equation}

The final step is to solve for the proportionality constant, $G$. As the relationship has been proven, any data point can be used to solve for this constant.
For simplicity, the control data point (used for setup) will be used, where $m_1$ = 100 kg, $m_2$ = 100 kg,  $r = x_2 - x_1 = 6\,\text{m} - 2\,\text{m} = 4\,\text{m}$.
We have
\[
F_g = G \left( \frac{m_1 m_2}{r^2} \right) 
= G \left( \frac{100\,\text{kg} \times 100\,\text{kg}}{4^2\,\text{m}^2} \right) 
= 4.17 \times 10^{-8}\,\text{N}
\]

\[
G = \frac{F_g \, r^2}{m_1 m_2} 
= \frac{(4.17 \times 10^{-8}\,\text{N}) (4\,\text{m})^2}{100\,\text{kg} \times 100\,\text{kg}}
= 6.67 \times 10^{-11}\,\text{N\,m}^2\text{/kg}^2
\].

Ergo, the data from the simulation can be used to derive NLUG and solve for $G$, the universal gravitational constant:


\begin{equation}
F_g = G \frac{m_1 m_2}{r^2}, \quad 
G = 6.67 \times 10^{-11}\,\text{N\,m}^2\text{/kg}^2
\label{eq:solvedNLUG}
\end{equation}

\section*{Error Analysis}

The first major source of error lies in the method of raising the angle to the right amount. The acceleration of the raising of the block must be zero; otherwise,
the block may start to slide earlier than it should have. Hence, the raising of the block is assumed to be the major source of error in this experiment.
Alongside this, the human reaction time to accurately read the protractor within decent tolerance is also a source of error.
Since the coefficients of static and kinetic friction were calculated using $\tan{\theta}$ and involved precise decimal values, 
inconsistency in the measured angles could result in either an overestimate or underestimate of the actual coefficients.

\section*{Discussion}

The physics concepts used in the lab are the coefficients of static and kinetic friction, which have major applications in the real world, specifically in materials science and engineering.
For example, understanding friction is crucial in designing systems like car wheels, where controlling the friction is vital to driving in different conditions (especially slippery roads).
Another such application of friction is in the design of screws, as the friction within threads is what allows them to hold materials together securely. Hence, a lower coefficient
of static friction would result in a looser screw, while a higher coefficient would result in a tighter screw, which can be crucial in construction, manufacturing, and architecture.
Understanding static and kinetic friction is not only important in product design but can be crucial to ensure safety in various applications, impacting everyday life. \\\\
Another way that this lab can be carried out is by using a spring scale to directly meassure the force needed to start moving the block and keep it moving at a constant velocity.
This method would also accurate measure the coefficients of static and kinetic friction, although it may be prone to different sources of error. A second method this lab can be carried out
is by using a motion sensor to track the block's movement down the incline, and similar to the cart and ramp lab, the motion sensor can be used to derive the acceleration (by differentiating the velocity data).
From the acceleration, the net force acting on the block can be calculated using Newton's Second Law, which can be used to calculate the frictional coefficients.

\section*{Conclusion}



Through this lab, it was determined that the coefficient of static friction ($\mu_s$) of wood on metal is roughly \textbf{0.33}, and the coefficient of kinetic friction ($\mu_k$) is roughly \textbf{0.27}. 
The coefficient of static friction can be compared to the established range $0.2-0.6$ for such surfaces.
The experimentally determined value falls within this range, indicating that the results are accurate and that the lab was completed successfully.
\end{document}
